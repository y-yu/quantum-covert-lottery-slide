% for notes environment
\usepackage{xsavebox}
\usepackage{hyperref}
\usepackage{graphicx}
\usepackage{luatexja}
\usepackage[hiragino-pro,deluxe,nfssonly,jis2004]{luatexja-preset}
\usepackage{fontspec}
\usepackage{epigraph}
\usepackage{etoolbox}
\usepackage{tikz}
\usepackage{framed}
\usepackage{mathtools}
\usepackage{listings}
\usepackage{libertine}
\usepackage[libertine]{newtxmath}
\usepackage{bxcoloremoji}
\usepackage{xcolor}
\usepackage{diagbox}
\usepackage{caption}
\usepackage{appendixnumberbeamer}
\usepackage{multirow}
\usepackage{xpatch}
\usepackage{multicol}
\usepackage{tabularx}
\usepackage{bxtexlogo}
\bxtexlogoimport{SATySFi}

\usetikzlibrary{fit}

\setmonofont{CMU Typewriter Text}

\definecolor{links}{HTML}{2A1B81}
\hypersetup{colorlinks,linkcolor=,urlcolor=links}

\usetheme{Boadilla}
\usecolortheme{seahorse}
% \usefonttheme{serif}


\xpatchcmd{\itemize}
  {\def\makelabel}
  {\ifnum\@itemdepth=1\relax
     \setlength\itemsep{1.2ex}% separation for first level
   \else
     \ifnum\@itemdepth=2\relax
       \setlength\itemsep{0.8ex}% separation for second level
       \setlength\topsep{1.2ex}
     \else
       \ifnum\@itemdepth=3\relax
         \setlength\itemsep{0.05ex}% separation for third level
         \setlength\topsep{0.8ex}
   \fi\fi\fi\def\makelabel
  }
 {}
 {}

\setbeamercolor{page number in head/foot}{bg=blue!10}
\setbeamertemplate{footline}{%
  \leavevmode%
  \hbox{%
    \begin{beamercolorbox}[wd=.3\paperwidth,ht=2.25ex,dp=1ex,center]{author in head/foot}%
      \usebeamerfont{author in head/foot}\insertshortauthor\hspace*{1ex}(\insertshortinstitute)
    \end{beamercolorbox}%
    \begin{beamercolorbox}[wd=.2\paperwidth,ht=2.25ex,dp=1ex,center]{title in head/foot}%
      \usebeamerfont{title in head/foot}\insertshorttitle
    \end{beamercolorbox}%
    \begin{beamercolorbox}[wd=.4\paperwidth,ht=2.25ex,dp=1ex,center]{date in head/foot}%
      \insertshortdate{} @ \InsertConference
    \end{beamercolorbox}%
    \begin{beamercolorbox}[wd=.1\paperwidth,ht=2.25ex,dp=1ex,center]{page number in head/foot}%
      \insertframenumber{} / \inserttotalframenumber\hspace*{1ex}
    \end{beamercolorbox}}%
  \vskip0pt%
}

\beamertemplatenavigationsymbolsempty

\setbeamertemplate{bibliography item}{\insertbiblabel}
\setbeamersize{description width=1cm}
\setbeamertemplate{items}[circle]
\setbeamertemplate{section in toc}[circle]
\setbeamertemplate{subsection in toc}{%
  \leavevmode\leftskip=2em
  {%
    \usebeamerfont*{itemize item}%
    \usebeamercolor{subsection number projected}%
    \color{bg}%
    \raise1.25pt\hbox{\donotcoloroutermaths$\bullet$}}%
  \hskip1.5ex\inserttocsubsection\par}

% Definitions for the title page
\newcommand*{\GitHub}[1]{%
  \gdef\InsertGitHub{#1}%
}
\newcommand*{\Email}[1]{%
  \gdef\InsertEmail{\href{mailto:#1}{#1}}%
}
\newcommand*{\Conference}[1]{%
  \gdef\InsertConference{#1}%
}
\setbeamerfont{title}{size=\huge, series=\bfseries, family=\mcfamily\rmfamily}
\setbeamercolor{title}{bg=white}
\setbeamerfont{subtitle}{size=\small, series=\mdseries, family=\mcfamily\rmfamily}%\gtfamily\sffamily}
\setbeamerfont{email}{size=\scriptsize, family=\ttfamily}
\setbeamercolor{email}{bg=white}
\setbeamerfont{date}{shape=\itshape, family=\rmfamily}
\setbeamerfont{vc}{size=\scriptsize, family=\ttfamily}
\setbeamercolor{vc}{bg=white}

\renewcommand{\figurename}{Fig}

\input{vc.tex}

\setbeamertemplate{title page}
{%
  \vbox{}
  \vfill
  \begingroup
    \centering
    \hrulefill\par%
    \vskip1ex\par%
    \begin{beamercolorbox}[sep=0pt,center,shadow=false,rounded=true]{title}
      \vfill
      \usebeamerfont{title}\inserttitle\par%
      \ifx\insertsubtitle\@empty%
      \else%
        \vskip0.5ex%
        {\usebeamerfont{subtitle}\usebeamercolor[fg]{subtitle}\insertsubtitle\par}%
      \fi%
      \vfill  
    \end{beamercolorbox}%
    \hrulefill\par%
    \vskip2ex%
    \begin{beamercolorbox}[sep=0pt,center,shadow=false,rounded=true]{author}
      \usebeamerfont{author}\insertauthor
    \end{beamercolorbox}
    \begin{beamercolorbox}[sep=0pt,center,shadow=false,rounded=true]{email}
      \usebeamerfont{email}\InsertEmail
    \end{beamercolorbox}
    \vskip0.1ex
    \begin{beamercolorbox}[sep=5pt,center,shadow=false,rounded=true]{institute}
      \usebeamerfont{institute}\insertinstitute
    \end{beamercolorbox}
    \begin{beamercolorbox}[sep=5pt,center,shadow=false,rounded=true]{date}
      \usebeamerfont{date}\insertdate \normalfont @ \InsertConference
    \end{beamercolorbox}
    \begin{beamercolorbox}[sep=0pt,center,shadow=false,rounded=true]{vc}
      \usebeamerfont{vc}
      \url{https://github.com/\InsertGitHub} (\texttt{\GITAbrHash})
    \end{beamercolorbox}
    % {\centering
    %   \href{https://creativecommons.org/licenses/by-nc/4.0/}{%
    %     \includegraphics[width=0.1\textwidth]{img/by-nc.pdf}%
    %   }%
    % }
    {\usebeamercolor[fg]{titlegraphic}\inserttitlegraphic\par}
  \endgroup
  \vfill
}
\setbeamertemplate{blocks}[rounded][shadow=false]

% ============ ここを消すとNote消える ================
% \mode<handout>{%
%   \usepackage{pgfpages}
%   \setbeameroption{show notes on second screen=right}
%   \setbeamertemplate{note page}{%
%     \vspace{2ex}\insertnote%
%   }
% }
% ============ ここを消すとNote消える ================


\renewcommand{\kanjifamilydefault}{\gtdefault}

\setbeamertemplate{caption}[numbered]
\resetcounteronoverlays{lstlisting}
\definecolor{bluegray}{rgb}{0.4, 0.6, 0.8}
\DeclareCaptionFormat{listing}{{\color{bluegray}\lstlistingname}#2#3}
\captionsetup[lstlisting]{format=listing, font={footnotesize}}
\captionsetup[figure]{name={図}}
\captionsetup[table]{name={表}}
\setbeamerfont{footnote}{size=\scriptsize}

\setmonofont[Ligatures=TeX]{CMU Typewriter Text}

\setbeamertemplate{items}[circle]

\input{./lib/quotebox.tex}
\input{./lib/footnotemark.tex}
\input{./lib/ballon.tex}
\input{./lib/callout.tex}
\input{./lib/listings.tex}
\input{./lib/notes.tex}
\input{./lib/stack.tex}
\input{./lib/card.tex}

\newcommand\ce[1]{%
  \coloremojiucs{#1}
}

\newcommand*{\lstitem}[1]{
  \setbox0\hbox{\lstinline{#1}}
  \item[\usebox0]
}

\presetkeys{todonotes}{inline, noinlinepar}{}

\renewcommand{\arraystretch}{1.2}
\newcolumntype{Y}{>{\centering\arraybackslash}X}

\title[Quantum Covert Lottery]{%
  Quantum Covert Lottery
}
\subtitle{高速化ではない量子コンピュータの応用}
\author[吉村 優]{%
  吉村 優(\textsc{Yoshimura} Hikaru)
}
\Email{hikaru\_yoshimura@r.recruit.co.jp}
\date[October 7-9, 2023]{%
  \oldstylenums{October 7-9, 2023}
}
\Conference{第56回 情報科学若手の会}
\institute[\InsertEmail]{%
  株式会社リクルート(Recruit Co., Ltd) \\
  \includegraphics[width=3cm]{./img/6_Brandlogo_2_Color.jpg}
}
\GitHub{y-yu/quantum-covert-lottery-slide}

\newcommand{\facesize}{1cm}
\newcommand\alicecallout[2]{
  \simplecallout[{\includegraphics[width=\facesize]{./img/alice_face.png}}]{#1}{cyan!10}{#2}
}
\newcommand\bobcallout[2]{
  \simplecallout[{\includegraphics[width=\facesize]{./img/bob_face.png}}]{#1}{orange!10}{#2}
}

\begin{document}

\frame{\maketitle}

\begin{frame}
  \frametitle{自己紹介}
  
  \begin{columns}
    \begin{column}{0.3\textwidth}
      \begin{center}
        \begin{figure}
          \includegraphics[width=0.95\textwidth]{img/bird2x.png}
        \end{figure}
      \end{center}
 
      \begin{table}[h]
        \begin{tabular}{ll}
          Twitter & \href{https://twitter.com/\_yyu\_}{@\_yyu\_} \\
          GitHub &  \href{https://github.com/y-yu}{y-yu} \\
        \end{tabular}
      \end{table}
    \end{column}
    \begin{column}{0.7\textwidth}
      \begin{itemize}
        \item 筑波大学 情報学群 情報科学類卒(2011-15,学士)
        \begin{itemize}
          \item プログラム論理研究室でシステムの研究
        \end{itemize}
        
        \item ドワンゴ ニコニコ動画 アカウントチーム

        \item スタディサプリENGLISH バックエンド(Scala)

        \item 未踏ターゲット2018(ゲート式量子コンピュータ)

        \item CTF(\url{https://urandom.team/})
        \begin{itemize}
          \item SECCON CTF 2022で世界57位(国内20位)
        \end{itemize}

        \item プログラミング
        \begin{itemize}
          \item Scala, \LaTeX, Rust, Go, Swift
          \item \SATySFi のバージョン\texttt{0.1.0}待ってます!\ce{:pray:}    
        \end{itemize}
      \end{itemize}
    \end{column}
  \end{columns}
\end{frame}

\begin{frame}
  \frametitle{目次}

  \tableofcontents
\end{frame}

\section{Covert Lotteryとは?}

\begin{frame}
  \frametitle{Covert Lotteryとは?}

  \begin{itemize}
    \item \emph{Covert Lottery}は\cite{covert_lottery_2021}で提案された、ちょっと変わった抽選
  \end{itemize}

  \begin{shadequote}[r]{}
    \begin{center}
      参加者2人が1bit(= $0$ or $1$)のいずれかの希望があるとき、
      \begin{enumerate}
        \item 二人の希望が別々であれば、そのとおりになる
        \item 衝突していたらランダムにする
      \end{enumerate}
    \end{center}
  \end{shadequote}

  \simplecallout[{\Large\ce{:thinking:}}]{+}{cyan!10}{いったい何に使えるのか?}
\end{frame}

\begin{frame}
  \frametitle{奢り・割勘問題}

  \begin{figure}[h]
    \includegraphics[width=0.65\textwidth]{./img/twitter.png}\cite{Y_N_Hoshi}
  \end{figure}
\end{frame}

\begin{frame}
  \frametitle{奢り・割勘問題}

  \begin{shadequote}[r]{}
    \begin{center}
      アリスとボブの飲食費について下記のいずれにするか決定する問題
      \begin{enumerate}
        \item ボブが全額を奢る
        \item 割勘とする
      \end{enumerate}
    \end{center}
  \end{shadequote}

  \begin{columns}
    \begin{column}{0.5\textwidth}
      \centering
      \emph{アリス(Alice)}

      \begin{figure}[h]
        \includegraphics[height=0.4\textheight]{img/alice.png}
      \end{figure}
    \end{column}
   
    \begin{column}{0.5\textwidth}
      \centering
      \emph{ボブ(Bob)}

      \begin{figure}[h]
        \includegraphics[height=0.4\textheight]{img/bob.png}
      \end{figure}
    \end{column}
  \end{columns}
\end{frame}

\section{古典Covert Lottery}

\begin{frame}
  \frametitle{カードを用いた古典Covert Lottery}
  
  \pause
  \begin{columns}
    \begin{column}{0.6\textwidth}
      \begin{enumerate}
        \item アリス・ボブに2枚のカード\heartcard,\clubcard を配る\footnote{%
          これらのカードはトランプのようにいずれも裏が\commitedcard となっており、
          裏向きになった状態でどちらのカードなのか特定することができない
        }
        \item アリス・ボブは表\ref{tbl:card_meaning}に従って
        希望を裏向き\commitedcard にして提出する\label{enum:cards_commited}

        \item \ballref{enum:cards_commited}で提出されたカードをシャッフルする
        
        \item どちらか1枚をドローして表向きにする \label{enum:result}
      \end{enumerate}

      \ballref{enum:result}のカードを表\ref{tbl:card_meaning}に対応させてプロトコルの結果とする
    \end{column}
    \begin{column}{0.4\textwidth}
      \begin{table}[h]
        \caption{カードの意味}
        \label{tbl:card_meaning}
        \begin{tabularx}{0.9\textwidth}{@{}| Y | Y |@{}}
          \hline
          カード & 意味 \\ \hline
          \heartcard & ボブの奢り \\ \hline
          \clubcard & 割勘 \\ \hline
        \end{tabularx}
      \end{table}
    \end{column}
  \end{columns}
\end{frame}

\begin{frame}
  \frametitle{ケーススタディ\ballcircle{1} --- 2人の希望が一致}

  \pause
  \begin{itemize}
    \item<+-> 2人の希望が一致しているので次のようなケース
      \begin{columns}
        \begin{column}{0.5\textwidth}
          \alicecallout{+}{\heartcard}
        \end{column}
        \begin{column}{0.5\textwidth}
          \bobcallout{-}{\heartcard}
        \end{column}
      \end{columns}

    \item<+-> これらをシャッフルして1枚選んだときは必ず\heartcard となる

    \item<+-> そしてこのときアリス・ボブは相手のカードについて
    \begin{itemize}
      \item 両方とも\heartcard だったのか
      \item 相手は\clubcard だったがランダムで\heartcard が選ばれたのか
    \end{itemize}
    \ce{:point_up:}のどちらなのか分からず、情報リークはない
  \end{itemize}
\end{frame}

\begin{frame}
  \frametitle{ケーススタディ\ballcircle{2} --- 2人の希望が衝突}

  \begin{itemize}
    \item<+-> 2人の希望が衝突しているので次のようなケース
      \begin{columns}
        \begin{column}{0.5\textwidth}
          \alicecallout{+}{\heartcard}
        \end{column}
        \begin{column}{0.5\textwidth}
          \bobcallout{-}{\clubcard}
        \end{column}
      \end{columns}

    \item<+-> これらをシャッフルしてランダムに選べば、結果は\heartcard,\clubcard それぞれ$\frac{1}{2}$の確率になる
    \begin{description}
      \item[結果が\heartcard]<+->\mbox{}\\
      \begin{itemize}
        \item アリスの希望通りとなるが、結果がボブの希望通りかランダムか不明
        \item ボブは衝突してアリスの希望\heartcard になったと特定
      \end{itemize}

      \item[結果が\clubcard]<+-> 同様
    \end{description}
  \end{itemize}
\end{frame}

\begin{frame}
  \frametitle{コイントスとAND計算のハイブリッドプロトコル}

  \bobcallout{+}{%
    期待値的にはボブが不公平なままだが、\\
    もし不本意に奢った場合はアリスのがめつさが分かる
  }

  \pause
  \alicecallout{-}{%
    このときアリスはボブの奢りが本意か\\
    不本意か分からないが、奢られを得る
  }
\end{frame}

\begin{frame}
  \frametitle{コイントスとAND計算のハイブリッドプロトコル}

  \begin{columns}
    \begin{column}{0.6\textwidth}
      \uncover<+->{
        \alicecallout{+}{%
          逆にアリスが不本意に\\
          割勘となってしまった場合、\\
          ボブの希望は割勘だと特定する
        }
      }

      \uncover<+->{
        \bobcallout{-}{%
          しかしこのときボブは\\
          アリスの希望が分からない
        }
      }
    \end{column}
    \begin{column}{0.4\textwidth}
      \uncover<+(-2)->{
        \begin{itemize}
          \item アリスが不本意に割勘となった場合、
          アリスは奢りを希望していたがボブは割勘を希望しており、
          ランダムで割勘となった

          \item<2-> このように希望通りになった側は相手の希望が分からず、
          希望通りにならかった側は相手の希望を知ることができる
        \end{itemize}
      }
    \end{column}
  \end{columns}
\end{frame}

\section{まとめ}

\begin{frame}
  \frametitle{まとめ}

  \pause
  \begin{itemize}
    \item<+-> 簡単なプロトコルで奢り・割勘問題に決着をつけられるかもしれない
    \begin{itemize}
      \item 2人の参加者は安目・高目と情報を賭博する
    \end{itemize}

    \item<+-> 今回紹介した技術は``Covert Lottery\cite{covert_lottery}''という名前が付いている
    \begin{itemize}
      \item Covert Lotteryを量子コンピューターでやるという記事\cite{quantum_covert_lottery}を過去に書いた
    \end{itemize}

    \item<+-> 今回は2人だったが、これを多人数拡張すると別のゲームに使えるかも\ce{:thinking_face:}
  \end{itemize}
\end{frame}

\section*{参考文献}
\begin{frame}[allowframebreaks]
  \frametitle{参考文献}
  \nocite{*}
  \bibliographystyle{junsrt_url}
  \bibliography{ref}
\end{frame}

\begin{frame}
  \centering
  {\Huge Thank you for the attention!}
\end{frame}

\end{document}
