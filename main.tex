\RequirePackage{luatex85}

% for notes environment
\usepackage{xsavebox}
\usepackage{hyperref}
\usepackage{graphicx}
\usepackage{luatexja}
\usepackage[hiragino-pro,deluxe,nfssonly,jis2004]{luatexja-preset}
\usepackage{fontspec}
\usepackage{epigraph}
\usepackage{etoolbox}
\usepackage{tikz}
\usepackage{framed}
\usepackage{mathtools}
\usepackage{listings}
\usepackage{libertine}
\usepackage[libertine]{newtxmath}
\usepackage{bxcoloremoji}
\usepackage{xcolor}
\usepackage{diagbox}
\usepackage{caption}
\usepackage{appendixnumberbeamer}
\usepackage{multirow}
\usepackage{xpatch}
\usepackage{multicol}
\usepackage{tabularx}
\usepackage{amsmath}
\usepackage{amsthm}
\usepackage{blochsphere}
\usepackage{bxtexlogo}
\usepackage[braket, qm]{qcircuit}
\bxtexlogoimport{SATySFi}

\usetikzlibrary{fit}

\setmonofont{CMU Typewriter Text}

\definecolor{links}{HTML}{2A1B81}
\hypersetup{colorlinks,linkcolor=,urlcolor=links}

\usetheme{Boadilla}
\usecolortheme{seahorse}
% \usefonttheme{serif}


\xpatchcmd{\itemize}
  {\def\makelabel}
  {\ifnum\@itemdepth=1\relax
     \setlength\itemsep{1.2ex}% separation for first level
   \else
     \ifnum\@itemdepth=2\relax
       \setlength\itemsep{0.8ex}% separation for second level
       \setlength\topsep{1.2ex}
     \else
       \ifnum\@itemdepth=3\relax
         \setlength\itemsep{0.05ex}% separation for third level
         \setlength\topsep{0.8ex}
   \fi\fi\fi\def\makelabel
  }
 {}
 {}

\setbeamercolor{page number in head/foot}{bg=blue!10}
\setbeamertemplate{footline}{%
  \leavevmode%
  \hbox{%
    \begin{beamercolorbox}[wd=.3\paperwidth,ht=2.25ex,dp=1ex,center]{author in head/foot}%
      \usebeamerfont{author in head/foot}\insertshortauthor\hspace*{1ex}(\insertshortinstitute)
    \end{beamercolorbox}%
    \begin{beamercolorbox}[wd=.2\paperwidth,ht=2.25ex,dp=1ex,center]{title in head/foot}%
      \usebeamerfont{title in head/foot}\insertshorttitle
    \end{beamercolorbox}%
    \begin{beamercolorbox}[wd=.4\paperwidth,ht=2.25ex,dp=1ex,center]{date in head/foot}%
      \insertshortdate{} @ \InsertConference
    \end{beamercolorbox}%
    \begin{beamercolorbox}[wd=.1\paperwidth,ht=2.25ex,dp=1ex,center]{page number in head/foot}%
      \insertframenumber{} / \inserttotalframenumber\hspace*{1ex}
    \end{beamercolorbox}}%
  \vskip0pt%
}

\beamertemplatenavigationsymbolsempty

\setbeamertemplate{bibliography item}{\insertbiblabel}
\setbeamersize{description width=1cm}
\setbeamertemplate{items}[circle]
\setbeamertemplate{section in toc}[circle]
\setbeamertemplate{subsection in toc}{%
  \leavevmode\leftskip=2em
  {%
    \usebeamerfont*{itemize item}%
    \usebeamercolor{subsection number projected}%
    \color{bg}%
    \raise1.25pt\hbox{\donotcoloroutermaths$\bullet$}}%
  \hskip1.5ex\inserttocsubsection\par}

% Definitions for the title page
\newcommand*{\GitHub}[1]{%
  \gdef\InsertGitHub{#1}%
}
\newcommand*{\Email}[1]{%
  \gdef\InsertEmail{\href{mailto:#1}{#1}}%
}
\newcommand*{\Conference}[1]{%
  \gdef\InsertConference{#1}%
}
\setbeamerfont{title}{size=\huge, series=\bfseries, family=\mcfamily\rmfamily}
\setbeamercolor{title}{bg=white}
\setbeamerfont{subtitle}{size=\small, series=\mdseries, family=\mcfamily\rmfamily}%\gtfamily\sffamily}
\setbeamerfont{email}{size=\scriptsize, family=\ttfamily}
\setbeamercolor{email}{bg=white}
\setbeamerfont{date}{shape=\itshape, family=\rmfamily}
\setbeamerfont{vc}{size=\scriptsize, family=\ttfamily}
\setbeamercolor{vc}{bg=white}

\renewcommand{\figurename}{Fig}

\input{vc.tex}

\setbeamertemplate{title page}
{%
  \vbox{}
  \vfill
  \begingroup
    \centering
    \hrulefill\par%
    \vskip1ex\par%
    \begin{beamercolorbox}[sep=0pt,center,shadow=false,rounded=true]{title}
      \vfill
      \usebeamerfont{title}\inserttitle\par%
      \ifx\insertsubtitle\@empty%
      \else%
        \vskip0.5ex%
        {\usebeamerfont{subtitle}\usebeamercolor[fg]{subtitle}\insertsubtitle\par}%
      \fi%
      \vfill  
    \end{beamercolorbox}%
    \hrulefill\par%
    \vskip2ex%
    \begin{beamercolorbox}[sep=0pt,center,shadow=false,rounded=true]{author}
      \usebeamerfont{author}\insertauthor
    \end{beamercolorbox}
    \begin{beamercolorbox}[sep=0pt,center,shadow=false,rounded=true]{email}
      \usebeamerfont{email}\InsertEmail
    \end{beamercolorbox}
    \vskip0.1ex
    \begin{beamercolorbox}[sep=5pt,center,shadow=false,rounded=true]{institute}
      \usebeamerfont{institute}\insertinstitute
    \end{beamercolorbox}
    \begin{beamercolorbox}[sep=5pt,center,shadow=false,rounded=true]{date}
      \usebeamerfont{date}\insertdate \normalfont @ \InsertConference
    \end{beamercolorbox}
    \begin{beamercolorbox}[sep=0pt,center,shadow=false,rounded=true]{vc}
      \usebeamerfont{vc}
      \url{https://github.com/\InsertGitHub} (\texttt{\GITAbrHash})
    \end{beamercolorbox}
    % {\centering
    %   \href{https://creativecommons.org/licenses/by-nc/4.0/}{%
    %     \includegraphics[width=0.1\textwidth]{img/by-nc.pdf}%
    %   }%
    % }
    {\usebeamercolor[fg]{titlegraphic}\inserttitlegraphic\par}
  \endgroup
  \vfill
}
\setbeamertemplate{blocks}[rounded][shadow=false]

% ============ ここを消すとNote消える ================
% \mode<handout>{%
%   \usepackage{pgfpages}
%   \setbeameroption{show notes on second screen=right}
%   \setbeamertemplate{note page}{%
%     \vspace{2ex}\insertnote%
%   }
% }
% ============ ここを消すとNote消える ================


\renewcommand{\kanjifamilydefault}{\gtdefault}

\setbeamertemplate{caption}[numbered]
\resetcounteronoverlays{lstlisting}
\definecolor{bluegray}{rgb}{0.4, 0.6, 0.8}
\DeclareCaptionFormat{listing}{{\color{bluegray}\lstlistingname}#2#3}
\captionsetup[lstlisting]{format=listing, font={footnotesize}}
\captionsetup[figure]{name={図}}
\captionsetup[table]{name={表}}
\setbeamerfont{footnote}{size=\scriptsize}

\setmonofont[Ligatures=TeX]{CMU Typewriter Text}

\setbeamertemplate{items}[circle]

\input{./lib/quotebox.tex}
\input{./lib/footnotemark.tex}
\input{./lib/ballon.tex}
\input{./lib/callout.tex}
\input{./lib/listings.tex}
\input{./lib/notes.tex}
\input{./lib/stack.tex}
\input{./lib/card.tex}

\newcommand\ce[1]{%
  \coloremojiucs{#1}
}

\newcommand*{\lstitem}[1]{
  \setbox0\hbox{\lstinline{#1}}
  \item[\usebox0]
}

\presetkeys{todonotes}{inline, noinlinepar}{}

\renewcommand{\arraystretch}{1.2}
\newcolumntype{Y}{>{\centering\arraybackslash}X}

\title[Quantum Covert Lottery]{%
  Quantum Covert Lottery
}
\subtitle{高速化ではない量子コンピュータの応用}
\author[吉村 優]{%
  吉村 優(\textsc{Yoshimura} Hikaru)
}
\Email{hikaru\_yoshimura@r.recruit.co.jp}
\date[October 7-9, 2023]{%
  \oldstylenums{October 7-9, 2023}
}
\Conference{第56回 情報科学若手の会}
\institute[\InsertEmail]{%
  株式会社リクルート(Recruit Co., Ltd) \\
  \includegraphics[width=3cm]{./img/6_Brandlogo_2_Color.jpg}
}
\GitHub{y-yu/quantum-covert-lottery-slide}

\newcommand{\facesize}{1cm}
\newcommand\alicecallout[2]{
  \simplecallout[{\includegraphics[width=\facesize]{./img/alice_face.png}}]{#1}{cyan!10}{#2}
}
\newcommand\bobcallout[2]{
  \simplecallout[{\includegraphics[width=\facesize]{./img/bob_face.png}}]{#1}{orange!10}{#2}
}


\newcommand{\Zero}{\left(
    \begin{array}{c}
      1 \\
      0
    \end{array}
  \right)}

\newcommand{\One}{\left(
    \begin{array}{c}
      0 \\
      1
    \end{array}
  \right)}

\newcommand{\天国}{\textcolor{cyan}{\ket{\text{天国}}}}
\newcommand{\地獄}{\textcolor{cyan}{\ket{\text{地獄}}}}
\newcommand{\生}{\textcolor{orange}{\ket{\text{生}}}}
\newcommand{\死}{\textcolor{orange}{\ket{\text{死}}}}

\begin{document}

\frame{\maketitle}

\begin{frame}
  \frametitle{自己紹介}
  
  \begin{columns}
    \begin{column}{0.3\textwidth}
      \begin{center}
        \begin{figure}
          \includegraphics[width=0.95\textwidth]{img/bird2x.png}
        \end{figure}
      \end{center}
 
      \begin{table}[h]
        \begin{tabular}{ll}
          Twitter & \href{https://twitter.com/\_yyu\_}{@\_yyu\_} \\
          GitHub &  \href{https://github.com/y-yu}{y-yu} \\
        \end{tabular}
      \end{table}
    \end{column}
    \begin{column}{0.7\textwidth}
      \begin{itemize}
        \item 筑波大学 情報学群 情報科学類卒(2011-15,学士)
        \begin{itemize}
          \item プログラム論理研究室で型システムの研究
        \end{itemize}

        \item スタディサプリENGLISH バックエンド(Scala)

        \item 未踏ターゲット2018(ゲート式量子コンピュータ)

        \item CTF(\url{https://urandom.team/})
        \begin{itemize}
          \item SECCON CTF 2023 Quals 72位(国内26位)
        \end{itemize}

        \item iOS・macOS向けコーヒー抽出支援アプリ\ce{:coffee:}

        \item プログラミング
        \begin{itemize}
          \item Scala, \LaTeX, Rust, Swift
          \item \SATySFi のバージョン\texttt{0.1.0}待ってます!\ce{:pray:}
        \end{itemize}
      \end{itemize}
    \end{column}
  \end{columns}
\end{frame}

\begin{frame}
  \frametitle{目次}

  \tableofcontents
\end{frame}

\section{Covert Lotteryとは?}

\begin{frame}
  \frametitle{Covert Lotteryとは?}

  \begin{itemize}
    \item \emph{Covert Lottery}は\cite{covert_lottery_2021}で提案された、ちょっと変わった抽選
  \end{itemize}

  \pause
  \begin{shadequote}[r]{}
    参加者2人が1bit(= $0$ or $1$)のいずれかの希望があるとき、
    \begin{enumerate}
      \item 二人の希望が一致していれば、それが採用される
      \item 衝突していたらランダムにする
    \end{enumerate}
  \end{shadequote}

  \pause
  \simplecallout[{\LARGE\ce{:thinking:}}]{+}{cyan!10}{{\LARGE いったい何に使えるのか?}}
\end{frame}

\begin{frame}
  \frametitle{%
    奢り・割り勘問題\footnote{%
      \cite{covert_lottery_2021}では将棋などの先攻・後攻を決める問題を例にしている。%
    }%
  }

  \begin{columns}
    \begin{column}{0.7\textwidth}
      \begin{figure}[h]
        \includegraphics[width=0.85\textwidth]{./img/twitter.png}\cite{Y_N_Hoshi}
      \end{figure}
    \end{column}
    \begin{column}{0.3\textwidth}
      \simplecallout[{\LARGE\ce{:sunglasses:}}]{+}{green!10}{{\large\ce{:point_left:}これか!?}}
    \end{column}
  \end{columns}
\end{frame}

\begin{frame}
  \frametitle{奢り・割り勘問題}

  \begin{shadequote}[r]{}
    アリスとボブの飲食費について下記のいずれにするか決定する問題
    \begin{enumerate}
      \item ボブが全額を奢る
      \item 割り勘とする
    \end{enumerate}
  \end{shadequote}

  \begin{columns}
    \begin{column}{0.5\textwidth}
      \centering
      \emph{アリス(Alice)}

      \begin{figure}[h]
        \includegraphics[height=0.35\textheight]{img/alice.png}
      \end{figure}
    \end{column}
   
    \begin{column}{0.5\textwidth}
      \centering
      \emph{ボブ(Bob)}

      \begin{figure}[h]
        \includegraphics[height=0.35\textheight]{img/bob.png}
      \end{figure}
    \end{column}
  \end{columns}
\end{frame}

\section{古典Covert Lottery}

\begin{frame}
  \frametitle{%
    カードを用いた古典Covert Lottery\protect\footnote{%
      \cite{covert_lottery_2021}では3人以上への拡張も踏まえてやや複雑な方法が説明されており、
      このプロトコルは発表者が2人を前提に独自に簡略化したものとなっている。%
    }
  }
  次のように物理的なカード\footnote{%
    これらのカードはトランプのようにいずれも裏が\commitedcard となっており、
    裏向きになった状態でどちらのカードなのか特定することができない。%
  }を用いて行う

  \pause
  \begin{columns}
    \begin{column}{0.6\textwidth}
      \begin{enumerate}
        \item アリス・ボブに2枚のカード\heartcard,\clubcard を配る
        \item アリス・ボブは表\ref{tbl:card_meaning}に従って
        希望を裏向き\commitedcard にして提出する\label{enum:cards_commited}

        \item \ballref{enum:cards_commited}で提出されたカードをシャッフルする
        
        \item どちらか1枚をドローして表向きにする \label{enum:result}
      \end{enumerate}

      \ballref{enum:result}のカードを表\ref{tbl:card_meaning}に対応させてプロトコルの結果とする
    \end{column}
    \begin{column}{0.4\textwidth}
      \begin{table}[h]
        \caption{カードの意味}
        \label{tbl:card_meaning}
        \begin{tabularx}{0.9\textwidth}{@{}| Y | Y |@{}}
          \hline
          カード & 意味 \\ \hline
          \heartcard & ボブの奢り \\ \hline
          \clubcard & 割り勘 \\ \hline
        \end{tabularx}
      \end{table}
    \end{column}
  \end{columns}
\end{frame}

\begin{frame}
  \frametitle{ケーススタディ\ballcircle{1} --- 2人の希望が一致}

  \pause
  \begin{itemize}
    \item<+-> 2人の希望が一致しているので次のようなケース
      \begin{columns}
        \begin{column}{0.5\textwidth}
          \alicecallout{+}{\heartcard}
        \end{column}
        \begin{column}{0.5\textwidth}
          \bobcallout{-}{\heartcard}
        \end{column}
      \end{columns}

    \item<+-> これらをシャッフルして1枚選んだときは必ず\heartcard となる

    \item<+-> 2人の希望が一致していればこのように必ずそちらが選ばれる
  \end{itemize}
\end{frame}

\begin{frame}
  \frametitle{ケーススタディ\ballcircle{2} --- 2人の希望が衝突}

  \begin{itemize}
    \item<+-> 2人の希望が衝突しているので次のようなケース
      \begin{columns}
        \begin{column}{0.5\textwidth}
          \alicecallout{+}{\heartcard}
        \end{column}
        \begin{column}{0.5\textwidth}
          \bobcallout{-}{\clubcard}
        \end{column}
      \end{columns}

    \item<+-> これらをシャッフルしてランダムに選べば、結果は\heartcard,\clubcard それぞれ$\frac{1}{2}$の確率になる
    \begin{description}
      \item[結果が\heartcard]<+-> アリスの希望どおり
      \item[結果が\clubcard]<+-> ボブの希望どおり
    \end{description}
    このように2つの結果がそれぞれ50\%のランダムとなる
  \end{itemize}
\end{frame}

\section{量子コンピュータとシュレディンガーの猫}

\begin{frame}
  \frametitle{量子コンピュータとシュレディンガーの猫}

  \begin{columns}
    \begin{column}{0.55\textwidth}
      \begin{minipage}[t][.6\textheight][t]{\textwidth}
        \tableofcontents[currentsection]
      \end{minipage}
    \end{column}
    \begin{column}{0.45\textwidth}
      \begin{itemize}
        \item このCovert Lotteryをカードではなくて量子コンピュータでやりたい

        \item その前にまずは量子コンピュータの基礎的なところを解説
      \end{itemize}
    \end{column}
  \end{columns}
\end{frame}

\begin{frame}
  \frametitle{量子コンピュータ}

  \begin{itemize}
    \item 古典コンピュータは1bitで\heartcard,\clubcard のような2つの値しか持たない

    \item 一方で量子コンピュータの1bitに相当する\emph{1qubit}(量子ビット)は
    2つの複素数$c_0, c_1$によって式\ref{eq:qubit}のように拡張される
    \begin{align}
      c_0\ket{\heartcard} + c_1\ket{\clubcard} \label{eq:qubit}
    \end{align}

    \item 古典コンピュータの1bitと同じ$\ket{\heartcard}, \ket{\clubcard}$のとき、
    $c_0, c_1$はそれぞれ次のようになる
    \begin{description}
      \item[$\ket{\heartcard}$] $c_0 = 1, c_1 = 0$
      \item[$\ket{\clubcard}$] $c_0 = 0, c_1 = 1$
    \end{description}

    % \item たとえば$\ket{\heartcard} = \Zero, \ket{\clubcard} = \One$のように行列で表せる
    % \[
    %   c_0\Zero + c_1\One = \left(
    %     \begin{array}{c}
    %       c_0 \\
    %       c_1
    %     \end{array}
    %   \right)
    % \]
  \end{itemize}
\end{frame}

\begin{frame}
  \frametitle{量子コンピュータ}

  \[
    c_0\ket{\heartcard} + c_1\ket{\clubcard} \tag{\ref{eq:qubit}}
  \]
  \begin{itemize}
    \item 式\ref{eq:qubit}の複素数$c_0, c_1$は\textbf{確率振幅}と呼ばれ、
    次のように$\ket{\heartcard}, \ket{\clubcard}$が観測される確率を得ることができる
    \begin{description}
      \item[$\ket{\heartcard}$が観測される確率] $|c_0|^2$
      \item[$\ket{\clubcard}$が観測される確率] $|c_1|^2$
    \end{description}

    \item 確率なので、$c_0, c_1$は次の条件式\ref{eq:probability_amplitude_constraint}を満す
    \begin{align}
      |c_0|^2 + |c_1|^2 = 1 \label{eq:probability_amplitude_constraint}
    \end{align}

    \item 一方で$c_0, c_1$の具体的な値を直接知る方法はない
  \end{itemize}
\end{frame}

\begin{frame}
  \frametitle{ブロッホ球}

  \begin{columns}
    \begin{column}{0.6\textwidth}
      \begin{itemize}
        \item 複素数は実数$a, b$を用いて$a + b\sqrt{-1}$のように表現される
    
        \item 1qubitの表現に2つの複素数$c_0, c_1$が必要なので、4変数の自由度があるが
        下記2つの条件により\textbf{球の表面座標}と考えることができる
        \begin{enumerate}
          \item 確率の満す条件式\ref{eq:probability_amplitude_constraint}
          \item $c_0$が実数になるように$c_1$を調整してもいい
          (同じとみなせるqubitが存在する)
        \end{enumerate}

        \item この球のことを\textbf{ブロッホ球}と呼び、
        たとえば$\ket{\heartcard}$や$\ket{\clubcard}$はそれぞれ
        球の北極と南極の座標に対応する
      \end{itemize}
    \end{column}
    \begin{column}{0.4\textwidth}
      \begin{figure}
        \begin{blochsphere}[radius=0.4\textwidth, tilt=15,rotation=-20,opacity=0.02]
          \drawBallGrid[style={opacity=0.1}]{40}{40}
      
          \drawAxis[style={cyan}]{0}{0};
          
          \labelLatLon{up}{90}{0};
          \labelLatLon{down}{-90}{90};
          \labelLatLon{left}{0}{0};
          \labelLatLon{right}{180}{0};
          \node[above] at (up) {$\ket{\heartcard}$};
          \node[below] at (down) {$\ket{\clubcard}$};
        \end{blochsphere}
        \caption{ブロッホ球}
      \end{figure}
    \end{column}
  \end{columns}
\end{frame}

\begin{frame}
  \frametitle{シュレディンガーの猫}

  \begin{columns}
    \begin{column}{0.4\textwidth}
      \begin{figure}
        \begin{blochsphere}[radius=0.38\textwidth, tilt=15,rotation=-20,opacity=0.02]
          \drawBallGrid[style={opacity=0.1}]{40}{40}
      
          \drawAxis[style={cyan}]{0}{0};
          \drawAxis[style={orange}]{90}{0};
          \drawCircle[style={red}]{0}{-90};
          
          \labelLatLon{up}{90}{0};
          \labelLatLon{down}{-90}{90};
          \labelLatLon{left}{0}{0};
          \labelLatLon{right}{180}{0};
          \node[above] at (up) {$\ket{\heartcard}$};
          \node[below] at (down) {$\ket{\clubcard}$};
          \node[above] at (left) {$\ket{+}$};
          \node[above] at (right) {$\ket{-}$};
        \end{blochsphere}
        \caption{ブロッホ球上の$\ket{\pm}$}
      \end{figure}
    \end{column}
    \begin{column}{0.6\textwidth}
      \begin{itemize}
        \item ``シュレディンガーの猫''で有名なように、量子ビットは$\ket{\heartcard}$と$\ket{\clubcard}$の
        ``混ざった''ような状態を表現できる

        \item たとえば$c_0 = \frac{1}{\sqrt{2}}, c_1 = \pm\frac{1}{\sqrt{2}}$として
        次のような2つの量子ビット$\ket{\pm}$を考える
        \begin{align*}
          \left\{\begin{array}{lcl}
            \ket{+} &\equiv& \frac{1}{\sqrt{2}}\left(\ket{\heartcard} + \ket{\clubcard}\right) \\
            \ket{-} &\equiv& \frac{1}{\sqrt{2}}\left(\ket{\heartcard} - \ket{\clubcard}\right)
          \end{array}\right.
        \end{align*}

        \item これらは$\ket{\heartcard}, \ket{\clubcard}$が観測される確率が
        いずれも$\left|\pm\frac{1}{\sqrt{2}}\right|^2 = \frac{1}{2}$になる
        %\begin{itemize}
        %  \item したがってシュレディンガーの猫はブロッホ球の赤道上の座標に相当する状態である
        %\end{itemize}
      \end{itemize}
    \end{column}
  \end{columns}
\end{frame}

\begin{frame}
  \frametitle{量子ビットの測定}

  \begin{columns}
    \begin{column}{0.2\textwidth}
      \begin{figure}
        \begin{blochsphere}[radius=0.6\textwidth, tilt=15,rotation=-20,opacity=0.02]
          \drawBallGrid[style={opacity=0.1}]{40}{40}
      
          \drawAxis[style={cyan}]{0}{0};
          \drawAxis[style={orange}]{90}{0};
          \drawCircle[style={red}]{0}{-90};
          
          \labelLatLon{up}{90}{0};
          \labelLatLon{down}{-90}{90};
          \labelLatLon{left}{0}{0};
          \labelLatLon{right}{180}{0};
          \node[above] at (up) {$\ket{\heartcard}$};
          \node[below] at (down) {$\ket{\clubcard}$};
          \node[above] at (left) {$\ket{+}$};
          \node[above] at (right) {$\ket{-}$};
        \end{blochsphere}
      \end{figure}
    \end{column}
    \begin{column}{0.8\textwidth}
      \begin{itemize}
        \item 量子ビットの確率振幅を直接知ることはできない

        \item 量子ビットの測定結果は測定に用いる\textbf{計算基底}によって変わる
        \begin{enumerate}
          \item ブロッホ球の赤道にある$\ket{+}$は
          %、測定の仕方によってたとえば次のように観測結果が異なる
          \begin{description}
            \item[$\left\{\ket{\heartcard}, \ket{\clubcard}\right\}$で測定した時]
            $\ket{\heartcard}, \ket{\clubcard}$のいずれかが$\frac{1}{2}$で観測される

            \item[$\left\{\ket{+}, \ket{-}\right\}$で測定した時]
            $\ket{+}$が確率100\%で観測される
          \end{description}

          \item 一方でブロッホ球の北極にある$\ket{\heartcard}$は
          \begin{description}
            \item[$\left\{\ket{\heartcard}, \ket{\clubcard}\right\}$で測定した時]
            $\ket{\heartcard}$が確率100\%で観測される

            \item[$\left\{\ket{+}, \ket{-}\right\}$で測定した時]
            $\ket{\pm}$のいずれかが$\frac{1}{2}$で観測される
          \end{description}
        \end{enumerate}

        \item 量子ビットを収束させる対象はこのように測定時に選ぶことができる
      \end{itemize}
    \end{column}
  \end{columns}
\end{frame}

\newcommand{\HGateFigure}{%
  \begin{blochsphere}[radius=0.4\textwidth, tilt=15,rotation=-20,opacity=0.05]
    \drawBallGrid[style={opacity=0.1}]{40}{40}
  
    \drawAxis[style={cyan}]{0}{0}
    %\drawAxis[style={orange}]{90}{0}
    \drawCircle[style={red}]{0}{-90};
    \drawAxis[style={orange}]{90}{90}

    \labelLatLon{n1}{45}{90}
    \labelLatLon{n2}{-45}{-90}
    \draw[->] (n2)--(n1);
    \drawSmallCircle[style={dashed, thick}]{45}{90}{45}
    
    \labelLatLon{x}{0}{90}
    \labelLatLon{y}{0}{0}
    \labelLatLon{z}{90}{90}
    \labelLatLon{down}{-90}{90};
    
    \node[above] at (z) {$\ket{\heartcard} = \ket{0}$};
    \node[below] at (down) {$\ket{\clubcard} = \ket{1}$};
    %\fill (z) ++(90:4ex) node {$Z$};
    %\node[above] at (y) {$Y$};
    \node[below left] at (x) {$\ket{+}$};
    %\fill (x) ++(180:2em) node {$X$};
    \node[above] at (n1) {$n$};
  \end{blochsphere}%
}

\begin{frame}
  \frametitle{1量子ビットのシミュレータ実験}

  \begin{columns}
    \begin{column}{0.6\textwidth}
      \begin{itemize}
        \item IBM Quantum Composer\footnote{\url{https://quantum-computing.ibm.com/composer/}}で実際にシミュレーションしてみる

        \item 初期値である$\ket{0}$から$\ket{+}$を作るために\textbf{量子ゲート}を使う

        \item $H$ゲート\footnote{``アダマールゲート''と読みます。}は
        図\ref{fig:hadamard_center}の$n$を中心に$\pi$回転させるので、
        $\ket{\heartcard}$が$\ket{+}$へ移る

        \item $\ket{+}$を$\left\{\ket{\heartcard}, \ket{\clubcard}\right\} = \left\{\ket{0}, \ket{1}\right\}$で測定する次のような回路をやってみる
        \begin{figure}
          \centering
          \scalebox{1.0}{
          \Qcircuit @C=1.0em @R=0.2em @!R {
            \lstick{\ket{0} = \ket{\heartcard}} & \gate{H} & \meter
          }}
        \end{figure}
      \end{itemize}
    \end{column}
    \begin{column}{0.4\textwidth}
      \begin{figure}
        \HGateFigure
        \caption{$H$ゲートの回転中心$n$}
        \label{fig:hadamard_center}
      \end{figure}
    \end{column}
  \end{columns}
\end{frame}

\begin{frame}
  \frametitle{1量子ビットのシミュレータ実験}

  \begin{columns}
    \begin{column}{0.5\textwidth}
      \begin{itemize}
        \item 結果は図\ref{fig:hgate_result}のように、
        $\ket{0} = \ket{\heartcard}$と$\ket{1} = \ket{\clubcard}$が
        $\frac{1}{2}$の確率でそれぞれ測定されている

        \item これで量子回路としてシュレディンガーの猫が完成\ce{:cat:}

        \item 同様に$H\ket{1} = \ket{-}$かつ$H\ket{-} = \ket{1}$となる
      \end{itemize}
    \end{column}
    \begin{column}{0.5\textwidth}
      \begin{figure}
        \includegraphics[width=0.95\textwidth]{./img/hgate_histogram.pdf}
        \caption{$H$ゲート回路の測定結果}
        \label{fig:hgate_result}
      \end{figure}
    \end{column}
  \end{columns}
\end{frame}

\begin{frame}
  \frametitle{1量子ビットのシミュレータ実験}

  \begin{columns}
    \begin{column}{0.4\textwidth}
      \begin{figure}
        \HGateFigure
      \end{figure}
    \end{column}
    \begin{column}{0.6\textwidth}
      \begin{itemize}
        \item $H$ゲートは$n$中心に$\pi$回転なので次のように2回やれば元の$\ket{0}$に戻る
        \begin{figure}
          \centering
          \scalebox{1.0}{
          \Qcircuit @C=1.0em @R=0.2em @!R {
            \lstick{\ket{0} = \ket{\heartcard}} & \gate{H} & \gate{H} & \meter
          }}
        \end{figure}
        \begin{figure}
          \includegraphics[width=0.95\textwidth]{./img/hgate_hgate_histogram.pdf}
        \end{figure}
      \end{itemize}
    \end{column}
  \end{columns}
\end{frame}

\begin{frame}
  \frametitle{他の量子ゲート\ballcircle{1} --- $X$ゲート}

  \begin{columns}
    \begin{column}{0.6\textwidth}
      \begin{itemize}
        \item $X$ゲートは図\ref{fig:x_gate_center}の$X$軸を中心に
        $\pi$回転させるゲート

        \item したがって次がなりたつ
        \begin{align*}
          X\ket{0} &= \ket{1} \\
          X\ket{1} &= \ket{0}
        \end{align*}

        \item 古典コンピュータの\texttt{NOT}ゲートと似ている
      \end{itemize}
    \end{column}
    \begin{column}{0.4\textwidth}
      \begin{figure}
        \begin{blochsphere}[radius=0.4\textwidth, tilt=15,rotation=-20,opacity=0.05]
          \drawBallGrid[style={opacity=0.1}]{40}{40}
        
          \drawAxis[style={cyan}]{0}{0}
          %\drawAxis[style={orange}]{90}{0}
          \drawCircle[style={red}]{0}{-90};
          %\drawAxis[style={orange}]{90}{90}
      
          \labelLatLon{n1}{0}{90}
          \labelLatLon{n2}{0}{-90}
          \draw[orange,->] (n2)--(n1);
          \drawSmallCircle[style={dashed, thick}]{90}{90}{0}
          
          \labelLatLon{x}{0}{90}
          \labelLatLon{y}{0}{0}
          \labelLatLon{z}{90}{90}
          \labelLatLon{down}{-90}{90};
          
          \node[above] at (z) {$\ket{\heartcard} = \ket{0}$};
          \node[below] at (down) {$\ket{\clubcard} = \ket{1}$};
          %\fill (z) ++(90:4ex) node {$Z$};
          %\node[above] at (y) {$Y$};
          \node[below left] at (x) {$\ket{+}$};
          \node[above right] at (n2) {$\ket{-}$};
          %\fill (x) ++(180:2em) node {$X$};
          \node[above] at (n1) {$X$};
        \end{blochsphere}
        \caption{$X$ゲートの回転中心}
        \label{fig:x_gate_center}
      \end{figure}
    \end{column}
  \end{columns}
\end{frame}

\begin{frame}
  \frametitle{他の量子ゲート\ballcircle{2} --- $Z$, $S$ゲート}

  \begin{columns}
    \begin{column}{0.6\textwidth}
      \begin{itemize}
        \item $Z$ゲートは図\ref{fig:z_gate_center}の$Z$軸を中心に
        $\pi$回転させるゲート

        \item $S$ゲートは$Z$軸を中心に$\frac{\pi}{2}$回転させるゲート

        \item 次がなりたつ
        \begin{align*}
          Z\ket{+} &= \ket{-} \\
          Z\ket{-} &= \ket{+} \\
          SS\ket{+} &= Z\ket{+} = \ket{-}
        \end{align*}

        \item $S\ket{\pm}$は$\left\{\ket{\heartcard}, \ket{\clubcard}\right\}$で次のように表現できる
        \begin{align*}
          S\ket{\pm} = \frac{1}{\sqrt{2}}\left(\ket{\heartcard} \pm i\ket{\clubcard}\right)
        \end{align*}
      \end{itemize}
    \end{column}
    \begin{column}{0.4\textwidth}
      \begin{figure}
        \begin{blochsphere}[radius=0.4\textwidth, tilt=15,rotation=-20,opacity=0.05]
          \drawBallGrid[style={opacity=0.1}]{40}{40}
        
          \drawAxis[style={blue}]{90}{0}
          \drawAxis[style={orange}]{90}{90}
      
          \labelLatLon{l}{0}{0}
          \labelLatLon{r}{180}{0}
          \labelLatLon{n1}{90}{0}
          \labelLatLon{n2}{-90}{0}
          \draw[cyan,->] (n2)--(n1);
          \drawSmallCircle[style={dashed, thick}]{0}{0}{0}
          
          \labelLatLon{x}{0}{90}
          \labelLatLon{y}{0}{0}
          \labelLatLon{z}{90}{90}
          \labelLatLon{down}{-90}{90};
          
          \labelLatLon{n3}{0}{-90}

          %\node[above] at (z) {$\ket{0}$};
          \node[below] at (down) {$\ket{1}$};
          %\fill (z) ++(90:4ex) node {$Z$};
          %\node[above] at (y) {$Y$};
          \node[below left] at (x) {$\ket{+}$};
          \node[above right] at (n3) {$\ket{-}$};
          \node[right] at (l) {$S\ket{+}$};
          \node[left] at (r) {$S\ket{-}$};
          %\fill (x) ++(180:2em) node {$X$};
          \node[above] at (n1) {$Z$};
        \end{blochsphere}
        \caption{$Z$, $S$ゲートの回転中心}
        \label{fig:z_gate_center}
      \end{figure}
    \end{column}
  \end{columns}
\end{frame}

\begin{frame}
  \frametitle{2量子ビットゲート}

  \begin{itemize}
    \item ここまでは1量子ビットゲートの話だった

    \item 古典コンピュータの\texttt{NAND}ゲートのように、
    量子コンピュータにも2量子ビットを入力に持つゲートが存在する
    %\begin{itemize}
    %  \item ただし\texttt{NAND}ゲートは2入力1出力のように
    %  出力が入力よりも少ない(非可逆)となっているが、
    %  量子コンピュータのゲートは可逆性が要請されているので
    %  入力と出力が常に同じ数となる
    %\end{itemize}
  \end{itemize}
\end{frame}

\begin{frame}
  \frametitle{$CZ$ゲート}

  \begin{itemize}
    \item $CZ$ゲートは2量子ビットを引数にとり
    \begin{enumerate}
      \item 1量子ビット目が$\ket{0}$であれば何もせず、
      \item 一方で1量子ビット目が$\ket{1}$であれば2量子ビット目に$Z$ゲートを作用させる
    \end{enumerate}
  \end{itemize}

  \simplecallout[{\LARGE\ce{:thinking:}}]{+}{orange!10}{1量子ビット目が$\ket{+}$だったら?}

  \simplecallout[{\LARGE\ce{:sunglasses:}}]{-}{green!10}{%
    \begin{minipage}{0.8\textwidth}
      1量子ビット目を測定して、
      \begin{description}
        \item[$\ket{0}$が観測される] なにも起きない
        \item[$\ket{1}$が観測される] 2量子ビット目に$Z$ゲートが作用する
      \end{description}
    \end{minipage}
  }
\end{frame}

\begin{frame}
  \frametitle{$CZ$ゲート}

  \begin{itemize}
    \item $CZ$ゲートを次のような量子回路\ref{fig:cz_gate}で確認してみる
    \begin{figure}
      \centering
      \scalebox{1.0}{
      \Qcircuit @C=1.0em @R=0.2em @!R {
        \lstick{\ket{0}_1} & \gate{H} & \ctrl{1} & \meter \\
        \lstick{\ket{0}_2} & \gate{H} & \gate{Z} & \gate{H} & \meter
      }}
      \label{fig:cz_gate}
    \end{figure}
    
    \item 2量子ビット目の$\ket{0}_2$について考える
    \begin{enumerate}
      \item $H$ゲートにより$\ket{+}$となる

      \item $CZ$ゲートによって次のように分岐
      \begin{enumerate}
        \item $Z$ゲートが作用しなければ$\ket{+}$のまま
        \item $Z$ゲートが作用すれば$\ket{-}$となる
      \end{enumerate}

      \item 最後に$H$ゲートを作用させるが、このとき
      \begin{enumerate}
        \item $\ket{+}$であれば$H\ket{+} = \ket{0}$となる
        \item $\ket{-}$であれば$H\ket{-} = \ket{1}$となる
      \end{enumerate}
    \end{enumerate}

    \item 1量子ビット目は$\ket{+}$なので、$\ket{0}, \ket{1}$のどちらかになるかは
    それぞれ確率$\frac{1}{2}$である
  \end{itemize}
\end{frame}

\begin{frame}
  \frametitle{$CZ$ゲート}

  \begin{columns}
    \begin{column}{0.5\textwidth}
      \begin{figure}
        \centering
        \scalebox{1.0}{
        \Qcircuit @C=1.0em @R=0.2em @!R {
          \lstick{\ket{0}_1} & \gate{H} & \ctrl{1} & \meter \\
          \lstick{\ket{0}_2} & \gate{H} & \gate{Z} & \gate{H} & \meter
        }}
      \end{figure}
    \end{column}
    \begin{column}{0.5\textwidth}
      \simplecallout[{\LARGE\ce{:point_left:}\ce{:thinking:}}]{-}{red!10}{つまりまとめると……}
    \end{column}
  \end{columns}

  \begin{enumerate}
    \item 確率$\frac{1}{2}$で1量子ビット目の測定結果が$\ket{0}$なら2量子ビット目も$\ket{0}$
    \item 確率$\frac{1}{2}$で1量子ビット目の測定結果が$\ket{1}$なら2量子ビット目も$\ket{1}$
  \end{enumerate}

  \simplecallout[{\LARGE\ce{:eyes:}}]{+}{cyan!10}{シミュレーターでやってみる}
\end{frame}

\begin{frame}
  \frametitle{$CZ$ゲートのシミュレーション結果}

  \begin{center}
    \begin{figure}
      \includegraphics[width=0.45\textwidth]{./img/cz_gate_histogram.pdf}
      \caption{%
        $CZ$ゲートを使った回路のシミュレーション結果%
        \footnote{図の最下位ビットが1量子ビット目、最上位ビットが2量子ビット目となる。}%
      }
    \end{figure}
  \end{center}
  \begin{itemize}
    \item このように\ce{:point_up:}シミュレーション結果は$\ket{00}$か$\ket{11}$が$\frac{1}{2}$となる\ce{:person_gesturing_ok:}
  \end{itemize}
\end{frame}

\section{量子テレポーテーション}

\begin{frame}
  \frametitle{量子テレポーテーション}

  \begin{itemize}
    \item 2つの量子ビットが$\ket{+}$のとき$CZ$ゲートの後で
    \begin{enumerate}
      \item $Z$軸の回転ゲートを作用させ \label{enum:teleportation_1}
      \item $H$ゲートを作用させ \label{enum:teleportation_2}
      \item 測定を行う \label{enum:teleportation_3}
    \end{enumerate}

    \item たとえば次のような回路\ref{fig:teleportation_circuit}を考える
    \begin{columns}
      \begin{column}{0.6\textwidth}
        \begin{figure}
          \centering
          \scalebox{1.0}{
          \Qcircuit @C=1.0em @R=0.2em @!R {
            \lstick{\ket{0}_1} & \gate{H} & \ctrl{1} & \gate{Z\ballref{enum:teleportation_1}} & \gate{H\ballref{enum:teleportation_2}} & \meter & \ballref{enum:teleportation_3} \\
            \lstick{\ket{0}_2} & \gate{H} & \gate{Z} & \qw & \meter
          }}
          \caption{$CZ$ゲートとテレポーテーション回路}
          \label{fig:teleportation_circuit}
        \end{figure}
      \end{column}
      \begin{column}{0.4\textwidth}
        \begin{figure}
          \includegraphics[width=0.8\textwidth]{./img/cz_teleportation_histogram.pdf}
          \caption{回路\ref{fig:teleportation_circuit}のシミュレーション結果}
        \end{figure}
      \end{column}
    \end{columns}
  \end{itemize}
\end{frame}

\begin{frame}
  \frametitle{量子テレポーテーション}

  \begin{columns}
    \begin{column}{0.6\textwidth}
      \begin{figure}
        \centering
        \scalebox{1.0}{
        \Qcircuit @C=1.0em @R=0.2em @!R {
          \lstick{\ket{0}_1} & \gate{H} & \ctrl{1} & \gate{Z\ballref{enum:teleportation_1}} & \gate{H\ballref{enum:teleportation_2}} & \meter & \ballref{enum:teleportation_3}  \gategroup{1}{5}{1}{4}{.4em}{--} \\
          \lstick{\ket{0}_2} & \gate{H} & \gate{Z} & \qw & \meter
        }}
      \end{figure}
    \end{column}
    \begin{column}{0.4\textwidth}
      \begin{figure}
        \includegraphics[width=0.6\textwidth]{./img/cz_teleportation_histogram.pdf}
      \end{figure}
    \end{column}
  \end{columns}

  \begin{itemize}
    \item 下記のように1量子ビット目の測定結果に応じて$X$ゲートの有無の差があるものの、
    $Z$ゲートと$H$ゲートが2量子ビット目へテレポートする
  \end{itemize}

  \begin{columns}
    \begin{column}{0.5\textwidth}
      \begin{figure}
        \scalebox{1.0}{
        \Qcircuit @C=1.0em @R=0.2em @!R {
          \lstick{\ket{0}_2} & \gate{H} & \gate{Z\ballref{enum:teleportation_1}} & \gate{H\ballref{enum:teleportation_2}} & \meter
        }}
        \caption{1量子ビット目の測定結果が$0$}
      \end{figure}
    \end{column}
    \begin{column}{0.5\textwidth}
      \begin{figure}
        \scalebox{1.0}{
        \Qcircuit @C=1.0em @R=0.2em @!R {
          \lstick{\ket{0}_2} & \gate{H} & \gate{Z\ballref{enum:teleportation_1}} & \gate{H\ballref{enum:teleportation_2}} & \gate{X} & \meter
        }}
        \caption{1量子ビット目の測定結果が$1$}
      \end{figure}
    \end{column}
  \end{columns}
\end{frame}

\begin{frame}
  \frametitle{量子テレポーテーション}

  \begin{itemize}
    \item テレポートなので1量子ビット目からは$Z, H$ゲートが消えてしまう
  \end{itemize}

  \begin{columns}
    \begin{column}{0.5\textwidth}
      \begin{figure}
        \scalebox{1.0}{
        \Qcircuit @C=1.0em @R=0.2em @!R {
          \lstick{\ket{0}_1} & \gate{H} & \ctrl{1} & \gate{H} & \meter \\
          \lstick{\ket{0}_2} & \gate{H} & \gate{Z} & \meter
        }}
        \caption{$H$ゲートだけのテレポート}
        \label{fig:h_gate_teleportation}
      \end{figure}
    \end{column}
    \begin{column}{0.5\textwidth}
      \begin{figure}
        \includegraphics[width=0.5\textwidth]{./img/h_gate_teleportation.pdf}
        \caption{回路\ref{fig:h_gate_teleportation}のシミュレーション結果}
        \label{fig:h_gate_teleportation_histogram}
      \end{figure}
    \end{column}
  \end{columns}

  \begin{itemize}
    \item 1量子ビット目は$H$ゲートが2回作用しているので、$CZ$ゲートがなければキャンセルして常に$\ket{0}$となるが、
    シミュレーション結果\ref{fig:h_gate_teleportation_histogram}では1量子ビット目は$\ket{0},\ket{1}$がランダムとなる
  \end{itemize}

  \simplecallout{-}{green!10}{量子コンピュータでは量子操作をテレポートさせることができる!}    
\end{frame}

\section{Quantum Covert Lottery}

\begin{frame}
  \frametitle{Quantum Covert Lottery}

  \begin{columns}
    \begin{column}{0.55\textwidth}
      \begin{minipage}[t][.6\textheight][t]{\textwidth}
        \tableofcontents[currentsection]
      \end{minipage}
    \end{column}
    \begin{column}{0.45\textwidth}
      \begin{itemize}
        \item 量子コンピュータではもっと色々なゲートがあるが、
        とりあえず量子コンピュータ上でCovert LotteryをやるにはここまでのゲートでOK\ce{:person_gesturing_ok:}
        \begin{itemize}
          \item $X, S, Z, H$ゲート
          \item $CZ$ゲート
        \end{itemize}
        
        \item ここからはこれまで説明した量子ゲートで``Quantum Covert Lottery''を作っていく
      \end{itemize}
    \end{column}
  \end{columns}
\end{frame}

\newcommand{\CovertTable}{%
  \begin{tabularx}{0.9\textwidth}{@{}| Y | Y |@{}}
    \hline
    希望 & 意味 \\ \hline
    $0$ & ボブの奢り \\ \hline
    $1$ & 割り勘 \\ \hline
  \end{tabularx}%
}

\begin{frame}
  \frametitle{プロトコル}

  \begin{enumerate}
    \item アリスは表\ref{tbl:covert_meaning}にしたがって希望$a \in \{0, 1\}$と、
    乱数$x \in \{0, 1\}$を生成する

    \item アリスは次のような回路\ref{fig:alice_circuit}で表現される3量子ビットを用意する
    \begin{itemize}
      \item $a = 1$であれば、1量子ビット目に$S$ゲートを作用させる
      \item $x = 1$であれば、3量子ビット目に$Z$ゲートを作用させる
    \end{itemize}

    \item 3量子ビットを量子通信回線でボブへ送信する
  \end{enumerate}

  \begin{columns}
    \begin{column}{0.5\textwidth}
      \begin{figure}
        \centering
        \scalebox{1.0}{
        \Qcircuit @C=1.0em @R=0.2em @!R {
          \lstick{a}         & \cw      & \cw             & \control\cw  & \\
          \lstick{\ket{0}_1} & \gate{H} & \ctrl{1}        & \gate{S}\cwx  & \qw \\
          \lstick{\ket{0}_2} & \gate{H} & \gate{Z}        & \ctrl{1} & \qw \\
          \lstick{\ket{0}_3} & \gate{H} & \gate{Z}\cwx[1] & \gate{Z} & \qw \\
          \lstick{x}         & \cw      & \control\cw
        }}
        \caption{アリスの用意する3量子ビット}
        \label{fig:alice_circuit}
      \end{figure}
    \end{column}
    \begin{column}{0.5\textwidth}
      \begin{table}[h]
        \caption{希望の意味}
        \label{tbl:covert_meaning}
        \CovertTable
      \end{table}
    \end{column}
  \end{columns}
\end{frame}

\begin{frame}
  \frametitle{プロトコル}

  \begin{columns}
    \begin{column}{0.6\textwidth}
      \begin{enumerate}
        \item ボブはアリスから3量子ビットを受けとる
        
        \item ボブは希望$b \in \{0, 1\}$を選び図\ref{fig:bob_circuit}のような量子操作を行う
        \begin{itemize}
          \item $b = 1$であれば、1量子ビット目に$S$ゲートを作用させる
        \end{itemize}

        \item ボブは1量子ビット目に$H$ゲートを作用させたあと測定を行う
      \end{enumerate}
    \end{column}
    \begin{column}{0.4\textwidth}
      \begin{table}[h]
        \CovertTable
      \end{table}
    \end{column}
  \end{columns}

  \begin{figure}
    \centering
    \scalebox{1.0}{
    \Qcircuit @C=1.0em @R=0.2em @!R {
      \lstick{b}         & \cw      & \cw       & \cw       & \control\cw \\
      \lstick{\ket{0}_1} & \gate{H} & \ctrl{1}  & \gate{S^?} & \gate{S}\cwx & \gate{H} & \meter \\%& \control \cw \\
      \lstick{\ket{0}_2} & \gate{H} & \gate{Z}  & \ctrl{1}  & \qw      & \qw    \\%& \gate{X}\cwx & \meter \\
      \lstick{\ket{0}_3} & \gate{H} & \gate{Z^?} & \gate{Z}  & \qw      & \qw    %& \gate{X}\cwx
    }}
    \caption{ボブが行う量子操作}
    \label{fig:bob_circuit}
  \end{figure}
\end{frame}

\begin{frame}
  \frametitle{プロトコル}

  \begin{columns}
    \begin{column}{0.5\textwidth}
      \simplecallout{-}{red!10}{これどういうこと?}    
    \end{column}
    \begin{column}{0.5\textwidth}
      \simplecallout{+}{blue!10}{まずは$\ket{0}_1$だけ整理してみる}
    \end{column}
  \end{columns}

  \begin{itemize}
    \item 1量子ビット目とアリス・ボブの希望に注目すると図\ref{fig:1st_qubit}のようになる
    \begin{figure}
      \centering
      \scalebox{1.0}{
      \Qcircuit @C=1.0em @R=0.2em @!R {
        \lstick{a}         & \cw      & \cw          & \control\cw \\
        \lstick{\ket{0}_1} & \gate{H} & \ctrl{2}     & \gate{S}\cwx & \gate{S}\cwx[1] & \gate{H} & \meter \\
        \lstick{b}         & \cw      & \cw          & \cw          & \control\cw \\
                           &          & \dstick{CZ}
      }}
      \vspace{2ex}
      \caption{1量子ビット目とアリス・ボブの希望}
      \label{fig:1st_qubit}
    \end{figure}

    \item これをアリス・ボブの希望$a, b$で場合わけして考える
  \end{itemize}
\end{frame}

\begin{frame}
  \frametitle{ケーススタディ\ballcircle{1} --- アリス・ボブの希望が一致}

  \begin{itemize}
    \item アリス・ボブの希望が一致するときというのは$a = b$なので次のようになる
    \begin{columns}
      \begin{column}{0.5\textwidth}
        \begin{figure}
          \centering
          \scalebox{1.0}{
          \Qcircuit @C=1.0em @R=0.2em @!R {
            \lstick{\ket{0}_1} & \gate{H} & \ctrl{1}    & \gate{H} & \meter \\
                               &          & \dstick{CZ}
          }}
          \vspace{2ex}
          \caption{両者がボブの奢り$a = 0$で一致}
          \label{fig:a_equal_0}
        \end{figure}
      \end{column}
      \begin{column}{0.5\textwidth}
        \begin{figure}
          \centering
          \scalebox{1.0}{
          \Qcircuit @C=1.0em @R=0.2em @!R {
            \lstick{\ket{0}_1} & \gate{H} & \ctrl{1} & \gate{S} & \gate{S} & \gate{H} & \meter \\
                               &          & \dstick{CZ}
          }}
          \vspace{2ex}
          \caption{両者が割り勘$a = 1$で一致}
          \label{fig:a_equal_1}
        \end{figure}
      \end{column}
    \end{columns}

    \item $SS = Z$により、次のように2量子ビット目へテレポートするゲートが決まる
    \begin{description}
      \item[$a = b = 0$] $H$ゲート(+ $X$ゲート)
      \item[$a = b = 1$] $H$ゲートと$Z$ゲート(+ $X$ゲート)
    \end{description}
  \end{itemize}

  \simplecallout{-}{yellow!10}{$S$ゲートが消えているのがポイント!}  
\end{frame}

\begin{frame}
  \frametitle{ケーススタディ\ballcircle{2} --- アリス・ボブの希望が不一致}

  \begin{itemize}
    \item 一方で、アリス・ボブの希望が一致しないのは$a \ne b$なので次のようになる
    \begin{figure}
      \scalebox{1.0}{
      \Qcircuit @C=1.0em @R=0.2em @!R {
        \lstick{\ket{0}_1} & \gate{H} & \ctrl{1}    & \gate{S}  & \gate{H} & \meter \\
                           &          & \dstick{CZ}
      }}
      \vspace{2ex}
      \caption{アリスは割り勘・ボブは奢り、またはアリスは奢り・ボブは割り勘}
      \label{fig:a_eq_1_b_eq_0}
    \end{figure}
  
    \item いずれも$S$ゲートと$H$ゲートが2量子ビット目へテレポートする
  \end{itemize}

  \simplecallout{+}{yellow!10}{$S$ゲートが2量子ビット目へテレポートされる!}  
\end{frame}

\begin{frame}
  \begin{figure}
    \centering
    \scalebox{1.0}{
    \Qcircuit @C=1.0em @R=0.2em @!R {
      \lstick{\ket{q_0}} & \gate{H} & \gate{S} & \ctrl{1} & \gate{S} & \gate{H} & \meter  & \control \cw \\
      \lstick{\ket{q_1}} & \gate{H} & \qw      & \gate{Z} & \ctrl{1} & \qw      & \qw     & \gate{X}\cwx & \meter & \rstick{\cdots\; c_1} \\
      \lstick{\ket{q_2}} & \gate{H} & \qw      & \qw      & \gate{Z} & \gate{H} & \qw     & \gate{X}\cwx & \meter & \rstick{\cdots\; c_2}
    }}
    \caption{量子Covert Lotteryの回路}
  \end{figure}
\end{frame}

\section{Covert Lotteryと情報リーク}

\begin{frame}
  \tableofcontents[currentsection]
\end{frame}

\begin{frame}
  \frametitle{コイントスとAND計算のハイブリッドプロトコル}

  \bobcallout{+}{%
    期待値的にはボブが不公平なままだが、\\
    もし不本意に奢った場合はアリスのがめつさが分かる
  }

  \pause
  \alicecallout{-}{%
    このときアリスはボブの奢りが本意か\\
    不本意か分からないが、奢られを得る
  }
\end{frame}

\begin{frame}
  \frametitle{コイントスとAND計算のハイブリッドプロトコル}

  \begin{columns}
    \begin{column}{0.6\textwidth}
      \uncover<+->{
        \alicecallout{+}{%
          逆にアリスが不本意に\\
          割り勘となってしまった場合、\\
          ボブの希望は割り勘だと特定する
        }
      }

      \uncover<+->{
        \bobcallout{-}{%
          しかしこのときボブは\\
          アリスの希望が分からない
        }
      }
    \end{column}
    \begin{column}{0.4\textwidth}
      \uncover<+(-2)->{
        \begin{itemize}
          \item アリスが不本意に割り勘となった場合、
          アリスは奢りを希望していたがボブは割り勘を希望しており、
          ランダムで割り勘となった

          \item<2-> このように希望通りになった側は相手の希望が分からず、
          希望通りにならかった側は相手の希望を知ることができる
        \end{itemize}
      }
    \end{column}
  \end{columns}
\end{frame}

\section{まとめ}

\begin{frame}
  \frametitle{まとめ}

  \pause
  \begin{itemize}
    \item<+-> 簡単なプロトコルで奢り・割り勘問題に決着をつけられるかもしれない
    \begin{itemize}
      \item 2人の参加者は安目・高目と情報を賭博する
    \end{itemize}

    \item<+-> 今回紹介した技術は``Covert Lottery\cite{covert_lottery}''という名前が付いている
    \begin{itemize}
      \item Covert Lotteryを量子コンピューターでやるという記事\cite{quantum_covert_lottery}を過去に書いた
    \end{itemize}

    \item<+-> 今回は2人だったが、これを多人数拡張すると別のゲームに使えるかも\ce{:thinking_face:}
  \end{itemize}
\end{frame}

\section*{参考文献}
\begin{frame}[allowframebreaks]
  \frametitle{参考文献}%

  \nocite{*}
  \bibliographystyle{junsrt_url}
  \bibliography{ref}
\end{frame}

\begin{frame}
  \centering
  {\Huge Thank you for the attention!}
\end{frame}

\end{document}
